\chapter{Power/EM I} \label{c4_forthchapter:cha}

\section{Electronic Circuits}
\subsection{A basic electronic circuit}
The most basic electronic circuit consist of a power supply 
(i.e. a battery) and an electrical load (any component consuming electric power) 
connected to it on one side and to the "ground" (the reference point from 
which voltages are measured) on the other side.
% here we should add a figure of this basic circuit, as shown on Yossi's presentation
The difference in the electric potential between the power supply and the ground
creates an electric current which flow through the load to the ground.

The difference in electric potential between two points is measured in Volts 
(usually denoted by \textbf{\textit{V}}). The amount of current flowing thru
the circuit at a given time is measured in Amperes (denoted by \textbf{\textit{A}}).
The electrical resistance of the load is a measure of its opposition to the flow 
of electric current through it. It is measured in Ohms (and denoted by \textbf{\textit{R}}).

% Ohm's law: V=I*R - we should add it inside a box
$V=I*R$
\newline
Ohm's law defines the relationship between the Voltage, Current and Resistance in a circuit:
The voltage is equal to the current multiplied by the resistance of the load.
Since in most of the circuits we are using, the voltage is fixed 
(defined by the characteristics of the power supply), a change in the resistance
of the circuit will cause a change in the current in the opposite direction.
This means we can measure the current over time in order to calculate the resistance.

A useful analogy for the relations between V, I and R is to imagine a fountain on a high mountain,
where the water flow down through a river to the sea. The difference in height between the fountain
and the sea is the Voltage, the width of the river can be thought of as the resistance,
and the flow of the water is the current.

$Power = Work / Time$
\newline
% Power consumption: P=I*V - we should add it inside a box
Power consumption: $P=I*V$
\newline
Electricity can be used to do various kinds of work:
\begin{itemize}
    \item Electromagnetic work (light a bulb, transmit a WiFi signal)
    \item Thermal work (heating)
    \item Mechanical work (spin a motor, vibrate a speaker)
    \item Chemical work (charging a battery)
    \item Computational work (store or load from memory, compute a value)
\end{itemize}

The power consumption of a device is the work it does divided by time.
It is measured in Watts (\textbf{\textit{W}}).
The power consumption can be calculated as current (\textbf{\textit{I}})
multiplied by Voltage (\textbf{\textit{V}}).
Power is consumed when it leaves the circuit.

\subsection{Current and Voltage dividers}
Before we take a look at two simple electronic circuits, we need to
introduce two additional terms:
A \textbf{short circuit} is a piece of wire with almost no resistance at all.
An \textbf{open circuit} is a circuit which doesn't allow any
current to pass through it.

\subsubsection{Connecting in serial}
% diagrams of connecting a short circuit and an open circuit in serial to the load.
If we connect a short circuit after the load, it will have no influence on it.
If we connect an open circuit after the load, it will increase the resistance
to a very high value, causing the current to become zero effectively.

\subsubsection{Connecting in parallel}
% diagrams of connecting a short circuit and an open circuit in parallel to the load.
If we connect an open circuit in parallel to the load, the current will flow only thru the load path,
so the current on the open circuit will be 0.
However, the voltage drop between both points of the open circuit will be the same as the drop between
the load sides.

If we connect a short circuit in parallel to the load, the current will "prefer" flowing thru it rather
than thru the load, so the current thru the load will be equal 0, while the current thru the short
circuit will be very high - by Ohm's law.
Since the cable is not a perfect conductor, some of the energy will be consumes in the form of
thermal work, so the cable will heat, and possibly melt and start a fire\dots

\section{Measuring Power Consumption}
\subsection{Ammeter}
% an icon of Ammeter symbol ("A" inside a circle) - like here: https://thenounproject.com/term/ammeter/96074/
% maybe also a good photo of such device
An Ammeter (from \textbf{Am}pere \textbf{Meter}) is a device capable of measuring the amount of
electric current going through it. It has very low resistance, so it doesn't interrupt the system
connected to it.

\subsubsection{Using Ammeter to measure power consumption}
We need to "cut" the wire connected to the load and connect both sides to the Ammeter.
Doing so will cause all current flowing through the load to pass through the Ammeter as well,
so we will be able to read the current at any given time.
In case we know the voltage (i.e. a 5V battery, a 220V power socket), we can compute the power
consumption: $P=I*V$

The problem: sometimes, we don't want (or simply can't) cut the circuit after the load in order
to connect an Ammeter.
% Yossi started talking about what to do if we don't want to cut, but the conclusion
% (after talking about connecting a Voltmeter in parallel) was we can't do it without cut...

\subsection{Voltmeter}
% an icon of Voltmeter symbol ("V" inside a circle) - like here: https://commons.wikimedia.org/wiki/File:Voltmeter_symbol.png
% maybe also a good photo of such device
A Voltmeter is a device capable of measuring the electrical potential (also called "Voltage drop")
between two points in an electric circuit.

By connecting a Voltmeter in parallel with a very small and accurate resistor, we can measure the
electric current by Ohm's law: $I=V/R$

Summary: we learnt what is Power Consumption and how we can measure it.
A very important fact is that \textbf{Power Consumption varies with time!}.
If we can find a relationship between the secret information we want to extract and
the power consumption, we can recover this information by measuring the power consumption over time.

% TO BE CONTINUED: 
% https://www.temi.com/editor/t/iT2GGbpWLF-dheqB2Wljm3PU5R9Yqyu1ynigOlYve96G2opOgaLbI27Ilw944MYylUBuF6U_xc5uZKfvlG3VhCKYswg?loadFrom=DeliveryEmail

% 2nd lecture
\subsection{Types of electronic components}
In general there are two types of elements in the circuit, the first one are Passive devices like a resistors (a passive two-terminal electrical component that implements electrical resistance as a circuit element), inductors (is a passive two-terminal electrical component that stores energy in a magnetic field when electric current flows through it), capacitors and diodes. And there are Active devices like transistors (a semiconductor device used to amplify or switch electronic signals and electrical power), amplifiers (an electronic device that can increase the power of a signal (a time-varying voltage or current)) and ICs. From our perspective, the Active devices are much more interesting for us (attackers) as they are using electricity in order to control electricity. One example can be an amplifier that has audio signal and power supply as inputs and it generates a greater audio signal as an output using the power supply. 
Another interesting active element for this course is a transistor. In an integrated circuits, there are a lot of active devices such as transistors  that if we can analyze their behavior we can learn about the data that they are processing. So when we look at the final consumption of these active elements, we can figure out some kind of secrets.
There are many kind of transistors and we will concentrate on understanding a certain type called Field-Effect Transistor.
% TODO: add a scheme of Field-Effect Transistor

\subsection{ Field-Effect Transistor }
The field-effect transistor (FET) is an electronic device which uses an electric field to control the flow of current. FETs are 3-terminalled devices, having a source, gate, and drain terminal. FETs control the flow of current by the application of a voltage to the gate terminal, which in turn alters the conductivity between the drain and source terminals.
In order to understand FET first we need to dive into the basics of semiconductors.
\subsection{ Semiconductors }
A semiconductor is a substance, usually a solid chemical element or compound, that can conduct electricity under some conditions but not others, making it a good medium for the control of electrical current. Its conductance varies depending on the current or voltage applied to a control electrode, or on the intensity of irradiation by infrared (IR), visible light, ultraviolet (UV), or X rays.
In general, transistors are made of semiconducting materials such as silicon. There are a conductor like copper or gold and there are insulators like plastic or glass. 
Silicon atom has three parts: neutrons (not relevant for our use), protons with the positive charge (heavy) and electrons that have the negative charge and they are small and dynamic. Silicon atom has four electrons in its outer orbital and when we have a crystal silicon those 4 electrons are set in place very nicely. It means that pure silicon is a very bad conductor as  conducting means that the electrons can move around and in this case they are very comfortable where they are.
% TODO: add silicon atom diagram
Metals can be good conductors of electricity as they have "free electrons" that can move easily from atom to atom, the electricity involves the flow of electrons. As all of the outer electrons in Silicon crystal are involved in perfect covalent bonds, they cannot move around. So, silicon crystal is nearly insulator and very little electricity will flow through it.
We can change the behavior of  the silicon and turn it into a conductor by doping it. In doping, we mix a small amount of an impurity into the silicon crystal.
There are two types of impurities:
•	N-type – where phosphorus or arsenic is added to the silicon in small quantities. They both have  five outer electrons, so one of them is out of place when they get into the silicon lattice. While having nothing to bond to, the fifth electron is free to move around. As electrons have a negative charge, this kind of impurity called N-type.
•	P-type - where boron or gallium is added to the silicon. They both have only three outer electrons. So, when we mixed them into the silicon lattice, there will be "holes" in the lattice where a silicon electron has nothing to bond to. The hole is looking for an electron from a neighbor atom and when that’s happens the hole is "moving". As the absence of an electron creates the effect of a positive charge, this kind of impurity called P-type.
\subsection{ How does Field-Effect Transistor work? }
In the Field-Effect Transistor there are (as shown in figure ..) n+ areas (N-type) and P area (P-type). The n+ area contains a lot of free electrons and the P area contains a lot of 'holes'. When no electricity is connected to the gate, the free electrons from the N+ are moving to the holes so there are no free electrons within the semiconductor itself. That means electrons can't move from the source to the drain i.e. open circuit. When electricity is connected to the gate it charge a lot of free electrons to it. The free electrons in the gate can't move to the silicon itself as there is an oxide layer between them, but is pushes the electrons in the silicon down to the body in a way there are holes between the source and the drain. That way electrons can move from the source to the drain freely and we have close circuit. 
\subsection{ NOT Gate }

